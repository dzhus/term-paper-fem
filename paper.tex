\documentclass[10pt]{article}
\usepackage[utf8x]{inputenc}
\usepackage[english,russian]{babel}
\usepackage{amsmath,amssymb}

\usepackage[top=2.71cm, bottom=2.71cm, left=6.28cm, right=3.14cm]{geometry}

\usepackage{noweb}
\noweboptions{nomargintag}

  % rich title
\usepackage{titling}

\numberwithin{equation}{section}

% Russian traditions
\renewcommand{\epsilon}{\varepsilon}
\renewcommand{\phi}{\varphi}
\renewcommand{\leq}{\leqslant}
\renewcommand{\geq}{\geqslant}
\newcommand{\intl}{\int\limits}
\usepackage{misccorr}

\usepackage{tikz}
\usetikzlibrary{calc,decorations.pathreplacing}
\tikzset{dot/.style={circle,fill=black,scale=0.5}}

\newcommand{\name}[1]{\textsc{#1}}
\newcommand{\BibEmph}{\name}
\newcommand{\neword}{\emph}
\newcommand{\program}[1]{{\tt #1}}
\newcommand{\fem}{\textsc{мкэ}}
\newcommand{\matr}[1]{[#1]}
\newcommand{\mul}{\cdot}
\newcommand{\figref}[1]{рис. \ref{#1}}

\newcommand{\node}[1]{$\phi_{#1}$}
\newcommand{\element}[1]{$e_{#1}$}
%\DeclareMathOperator{\det}{det}

% Minus for matrices
\newcommand{\mm}{\llap{$-$}}
\newcommand{\phm}{\phantom{-}}

% Subfigures
\usepackage{subfigure}
\renewcommand{\thesubfigure}{\asbuk{subfigure})~}

% Bib in TOC
\usepackage[numbib,nottoc]{tocbibind}

\usepackage[unicode,
pdftex, colorlinks, linkcolor=black, citecolor=black,
pdfauthor=Dmitry Dzhus]{hyperref}

\begin{document}

\author{Дмитрий Джус}
\title{Курсовая работа по теме \\
  \Huge{«Метод конечных элементов»}}
\pretitle{\begin{center}\LARGE}
  \posttitle{\par\end{center}\vskip 3pc}
\date{}
\maketitle
\thispagestyle{empty}

\clearpage
\tableofcontents

\clearpage
\section*{Предмет работы и цели}

Настоящая курсовая работа посвящена методу конечных элементов. Дано
вводное описание метода, рассмотрен стержневой конечный элемент.
Изложены процедуры использования \fem{} для задач статики и динамики.
Представлена компьютерная реализация метода на алгоритмическом языке.

Целью работы является знакомство с теоретическими основами \fem{} и
способами реализации метода на практике.

\clearpage
\section{Метод конечных элементов}

\subsection{Введение}
\label{sec:intro}

Метод конечных элементов предназначен для решения задач
прикладной физики, в том числе механики сплошных сред.

Суть \fem{} заключается в аппроксимации исследуемой непрерывной
величины моделью, состоящей из набора \neword{конечных элементов}, и
последующем сведении исходной краевой задачи к системе алгебраических
уравнений.

\begin{figure}[hb]
  \centering
  \begin{tikzpicture}
    % triangles
    \coordinate [] (a) at (0,0);
    \coordinate [] (b) at ($ (a) + (1, 1) $);
    \coordinate [] (c) at ($ (b) + (-1.5, 0.2) $);
    \coordinate [] (d) at ($ (a) + (-1.2, 0.3) $);
    \coordinate [label=below:$D$] (e) at ($ (a) + (-.5, -1) $);
    \coordinate [] (f) at ($ (a) + (.7, -0.2) $);
    \coordinate [] (g) at ($ (e) + (-1.5, 1.2) $);
    \coordinate [] (h) at ($ (g) + (0.5, 1.5) $);
    \draw (a) -- (b) -- (c) -- cycle;
    \draw (a) -- (c) -- (d) -- cycle;
    \draw (e) -- (a) -- (d) -- cycle;
    \draw (a) -- (b) -- (f) -- cycle;
    \draw (a) -- (e) -- (f) -- cycle;
    \draw (e) -- (d) -- (g) -- cycle;
    \draw (g) -- (d) -- (c) -- cycle;
    \draw (g) -- (h) -- (c) -- cycle;

    \draw[thick, smooth cycle] plot coordinates{(b) (c) (h) (g) (e) (f)};
  \end{tikzpicture}
  \caption{Двумерная задача}
  \label{fig:2d-domain}
\end{figure}


\begin{figure}[hb]
  \centering
  \begin{tikzpicture}
    \node [dot, label=above:$\phi_1$] (A) at (0, 0) {};
    \node [dot, label=right:$\phi_2$] (B) at ($ (A)+(1,-1) $) {};
    \node [dot, label=below:$\phi_3$] (C) at ($ (A)+(-.5,-1.2) $) {};
    \draw (A) -- (B) -- (C) -- (A);
  \end{tikzpicture}
  \caption{Треугольный конечный элемент}
  \label{fig:triangle}
\end{figure}


На \figref{fig:2d-domain} представлена двумерная область $D$, в
которой решается некоторая краевая задача для функции $\phi(x, y)$,
которая в данном примере аппроксимирована треугольными конечными
элементами. Изображённый на \figref{fig:triangle} конечный элемент
задаётся таким образом, что в нём выполняется соотношение
\begin{equation}
  \label{eq:triangle-shape}
  \phi(x,y) = N_1(x, y) \phi_1 + N_2(x, y) \phi_2 + N_3(x, y) \phi_3,
\end{equation}
где $\phi_1,\phi_2,\phi_3$ — значения функции в узлах элемента, а
$N_1, N_2, N_3$ — непрерывные \neword{функции формы}. В \fem{} в
качестве неизвестных выступают значения $\phi_i$ в узлах конечных
элементов, на которые разбита рассматриваемая область. Аналитический
вид функций формы известен и выбирается исходя из требований к
адекватности модели задаче.

Таким образом, \fem{} позволяет вычислить значение функции в любой
точке рассматриваемой области $D$ при известных узловых значениях.
Этим он отличается от метода конечных разностей
(см. \cite{vorozhtsov98}). Кроме того, \fem{} может применяться с
нерегулярными сетками.

Существуют различные типы конечных элементов с разным количеством
степеней свободы и свойствами. Конкретный тип элементов для задачи
выбирается из различных соображений. В некоторых случаях (например,
при рассмотрении ферм или трубопроводов), конструктивные элементы
совпадают с конечными элементами. В общем случае конечные элементы
выбираются так, чтобы обеспечить должную аппроксимацию, при
обеспечении наименьшего уровня вычислительной сложности.

В дальнейшем более подробно будет рассмотрен стержневой элемент и его
использование в \fem{}.

\subsection{Общая процедура \fem{}}

Применение метода к решению задач состоит из нескольких этапов:

\begin{enumerate}
\item Определение геометрии задачи;

\item Выбор типа используемых конечных элементов и задание их свойств;

\item Построение подходящей сетки для заданной геометрии;

\item Приложение граничных условий к модели, в результате чего
  получается система алгебраических уравнений;

\item Решение полученной системы, благодаря чему определяются значения
  изучаемой величины в узлах сетки.
\end{enumerate}

Полученное решение и известные свойства элементов используются для
последующего анализа: вычисления интересующих параметров, производных
от найденных значений.

Например, \fem{} может быть применён для изучения воздействия
известных внешних сил на некоторую сплошную конструкцию. Смещение
точки конструкции дискретизируется с помощью конечно-элементной сетки,
после чего накладываются граничные условия закрепления и нагрузок.
Решение получившейся системы уравнений даёт набор смещений в узлах
конечных элементов, после чего деформации и напряжения в конструкции
могут быть вычислены с учётом физических свойств введённых элементов.

\subsection{Стержневой элемент}
\label{sec:bar}

Стержень — простой конечный элемент, особенно удобный для
моделирования двух- и трёхмерных ферм. Формальное определение этого
конечного элемента основано на следующих утверждениях
(см. \cite{hutton04}):

\begin{enumerate}
\item Стержень прямой;

\item Материал стержня подчиняется закону Гука;

\item Силы прилагаются только к концам стержня;

\item Стержень может передавать только осевое воздействие.
\end{enumerate}

К характеристикам стержня относятся длина $L$, площадь осевого
сечения $A$, модуль упругости $E$. Рассматриваются зависимости осевых
перемещений $u = u(x)$, относительной деформации $\epsilon =
\epsilon(x)$, напряжения $\sigma = \sigma(x)$ от координаты точки на
оси стержня.

\begin{figure}[hb]
  \centering
  \begin{tikzpicture}
    \node [dot] (A) at (0, 0) {};
    \node [dot] (B) at ($ (A) + (5, 0) $) {};
    \coordinate (shift) at (0, 0.3);

    \node [draw, circle] (Ae) at (A) {};
    \node [draw, circle] (Be) at (B) {};
    \draw (Ae.north) -- (Be.north);
    \draw (Ae.south) -- (Be.south);

    \draw [|<->|] ($ (A) + (shift) $) -- node[above] {$L$} ($ (B) + (shift) $);
    \draw [|->] ($ (A) - (shift) $) -- node[below] {$u_1$} ++(1.5, 0);
    \draw [|->] ($ (B) - (shift) $) -- node[below] {$u_2$} ++(0.5, 0);
    \begin{scope}[>=latex]
      \draw [->, thick] ($ (A) - (0.7, 0) $) node[above] {$f_1$} 
      -- ($ (A) $); 
      \draw [->, thick] ($ (B) $) -- ++(1, 0) 
      node[above] {$f_2$} ;
    \end{scope}

  \end{tikzpicture}
  \caption{Стержневой элемент}
  \label{fig:bar}
\end{figure}


На \figref{fig:bar} изображён стержневой элемент длины $L$ со
смещениями в концах стержня $u_1$ и $u_2$. К концам приложены силы
$f_1$ и $f_2$.

Считается, что перемещение $u$ линейно изменяется вдоль оси стержня:
\begin{equation}
  \label{eq:bar-displace}
  u(x) = \left( 1 - \frac{x}{L} \right) u_1 + \left( \frac{x}{L}
  \right) u_2.
\end{equation}

Таким образом, для стержня функции формы имеют вид:

\begin{equation}
  \label{eq:bar-shape}
  \begin{aligned}
    N_1(x) &= 1 - \frac{x}{L},\\
    N_2(x) &= \frac{x}{L}.
  \end{aligned}
\end{equation}

Из теории упругости известно, что $\epsilon = \frac{du}{dx}$ и $\sigma
= E\epsilon$. С учётом приближения \eqref{eq:bar-displace} получаем:
\begin{equation}
  \label{eq:bar-strain}
  \epsilon = \frac{u_2-u_1}{L}.
\end{equation}
Соответствующая сила, действующая на стержень, выражается следующим
образом:
\begin{equation*}
  P = A\sigma = AE\epsilon = \frac{AE}{L}(u_2-u_1).
\end{equation*}

Уравнение статического равновесия стержневого элемента:
\begin{equation}
  \begin{cases}
    f_1 &= -\frac{AE}{L}(u_2-u_1),\\
    f_2 &= \phm\frac{AE}{L}(u_2-u_1).
  \end{cases}
\end{equation}

В матричном виде оно может быть записано так:
\begin{equation}
  \label{eq:bar-equil}
  \underbrace{
  \frac{AE}{L}
  \begin{bmatrix}
    \phm{}1 && \mm1\\
    -1 && 1
  \end{bmatrix}}_{\matr{K}}
  \begin{bmatrix}
    u_1\\
    u_2
  \end{bmatrix}
  =
  \begin{bmatrix}
    f_1\\
    f_2
  \end{bmatrix}
\end{equation}

Матрица $\matr{K}$, связывающая смещения узлов конечного элемента с
приложенными к нему силами, называется \neword{матрицей жёсткости}
данного элемента. 

Введём обозначение $k = \frac{AE}{L}$, с учётом которого матрица
жёсткости стержня принимает вид:
\begin{equation}
  \label{eq:bar-stiffness}
  \matr{k} =
  \begin{bmatrix}
    \phm{}k&& \mm k\\
    -k && k
  \end{bmatrix}.
\end{equation}

\clearpage
\section{Статическая задача}

\subsection{Системы координат и нумерация}
\label{sec:notation}

В дальнейшем будем выделять две системы координат —
\neword{локальную}, оси которой ориентированы в соответствии с
положением конечного элемента в пространстве, и \neword{глобальную},
оси которой направлены одинаково для всех конечных элементов модели.
На \figref{fig:element-cs} смещения узлов стержневого элемента в
плоскости приведены в двух системах. Узловые смещения в глобальных
координатах обозначаются большим буквами, в локальных — маленькими.

\begin{figure}[hb]
  \tikzset{>=latex}
  \centering
  \subfigure[Глобальная система]{
    \begin{tikzpicture}
      \node [dot] (1) at (0, 0) {};
      \node [dot] (2) at ($ (1) + (xyz polar cs:angle=50,radius=3) $) {};
      \draw[thick] (1) -- (2);
      
      \coordinate (c) at ($ (1)!.5!(2) $);
      \draw (c) -- ++(.7, 0);
      \draw ($ (c) + (.5, 0) $) arc (0:50:.5);
      \draw ($ (c) + (xyz polar cs:angle=25, radius=.65) $) node {$\theta$};

      \coordinate (x) at ($ (1) + (-0.5,2.5) $);
      \draw[dashed, ->] (x) -- ++(1,0) node[below] {$X$};
      \draw[dashed, ->] (x) -- ++(0,1) node[left] {$Y$};
      
      \draw[->] (1) -- node[left] {$U_2^e$} ++(0, 1.4) {};
      \draw[->] (1) -- node[below] {$U_1^e$} ++(-0.8, 0) {};
      \draw[->] (2) -- node[left] {$U_4^e$} ++(0, 1.2) {};
      \draw[->] (2) -- node[below] {$U_3^e$} ++(1.3, 0) {};
    \end{tikzpicture}}
  \hskip 1cm
  \subfigure[Локальная система]{
    \begin{tikzpicture}
      \node [dot] (1) at (0, 0) {};
      \node [dot] (2) at ($ (1) + (xyz polar cs:angle=50,radius=3) $) {};
      \draw[thick] (1) -- (2);

      \coordinate (x) at ($ (1) + (-0.5,2.5) $);
      \draw[dashed, ->] (x) -- ++(xyz polar cs:angle=50,radius=1)
      node[right] {$x$};
      \draw[dashed, ->] (x) -- ++(xyz polar cs:angle=140,radius=1)
      node[left] {$y$};
            
      \draw[->] (1) -- node[below left] {$v_1^e$}
      ++(xyz polar cs:angle=140, radius=1) {};
      \draw[->] (1) -- node[below right] {$u_1^e$}
      ++(xyz polar cs:angle=50, radius=1.1) {};
      \draw[->] (2) -- node[above right] {$v_2^e$}
      ++(xyz polar cs:angle=-40, radius=1.7) {};
      \draw[->] (2) -- node[above left] {$u_2^e$}
      ++(xyz polar cs:angle=50, radius=1.1) {};
    \end{tikzpicture}}
  \caption{Системы координат}
  \label{fig:truss}
\end{figure}


Ориентация стержня на плоскости характеризуется углом $\theta$ наклона
элемента к глобальной оси $X$.

Связь локальных и глобальных координат выражается матричным
соотношением:
\begin{equation}
  \label{eq:glob-loc}
  \begin{bmatrix}
    u_1^e\\
    v_1^e\\
    u_2^e\\
    v_2^e\\
  \end{bmatrix}
  =
  \underbrace{
  \begin{bmatrix}
    \phm\cos\theta & \sin\theta & 0 & 0 \\
    -\sin\theta & \cos\theta & 0 & 0 \\
    \phm0 & 0 & \cos\theta & \sin\theta \\
    \phm0 & 0 & \mm\sin\theta & \cos\theta
  \end{bmatrix}}_{R_0}
  \begin{bmatrix}
    U_1^e\\
    U_2^e\\
    U_3^e\\
    U_4^e\\
  \end{bmatrix}
\end{equation}

На практике конечно-элементная сетка задаётся перечеслением координат
всех её узлов и перечислением элементов, соединяющих различные узлы
сетки. Таким образом, каждый стержневой элемент определяется парой
координат своих узлов $(X_i, Y_i)$ и $(X_j, Y_j)$. Знание этих
координат позволяет вычислить длину каждого узла:
\begin{equation}
  \label{eq:coord-length}
  L = \sqrt{(X_j-X_i)^2+(Y_j-Y_i)^2}
\end{equation}

Компоненты матрицы $R_0$ для каждого элемента могут быть найдены из соотношений
\begin{equation}
  \label{eq:dir-cosines}
  \begin{aligned}
    \cos\theta &= \frac{X_j-X_i}{L},\\
    \sin\theta &= \frac{Y_j-Y_i}{L}.\\
  \end{aligned}
\end{equation}

Итак, матрица жёсткости стержня \eqref{eq:bar-stiffness} в глобальных
координатах может быть определена при данных параметрах $A, E$ и
координатах узлов стержня.

Поскольку в модели обычно одновременно рассматривается множество
конечных элементов, удобно ввести единую глобальную нумерацию для
узловых смещений. В дальнейшем смещение узла \node{k} в глобальном
направлении $X$ будем обозначать как $U_{2k-1}$, а в направлении $Y$ —
$U_{2k}$. Например, для стержня \element{n} с узлами \node{i} и \node{j} вектор
смещений
\begin{equation*}
  {U^e}^{[n]}=
  \begin{bmatrix}
    U_1^e & U_2^e & U_3^e & U_4^e
  \end{bmatrix}
\end{equation*}
в глобальной нумерации будет иметь следующие компоненты:
\begin{equation}
  \label{eq:global-numbering}
  \begin{bmatrix}
    U_{2i-1} & U_{2i} & U_{2j-1} & U_{2j}
  \end{bmatrix}.
\end{equation}
 
Таким образом, смещения $N$ узлов описываются вектором длины $2N$.
Аналогичные обозначения будем использовать для внешних сил,
приложенных к узлам.

Для перехода от локальной нумерации к глобальной для каждого элемента
\element{n} с узлами \node{i}, \node{j} введём \neword{вектор
  связности}, описывающий глобальные индексы компонент узловых
смещений данного элемента:
\begin{equation}
  \label{eq:glob-num-vector}
  L^{[n]} =
  \begin{bmatrix}
    2i-1 & 2i & 2j-1 & 2j
  \end{bmatrix}
\end{equation}

Тогда можно видеть, что $L_i^{[n]}$ есть индекс $i$-ой компоненты
вектора ${U^e}^{[n]}$ в векторе $U$, то есть выполняется соотношение
\begin{equation}
  \label{eq:num-vector-transform}
  U_{L_i^{[n]}} = {U_i^e}^{[n]}.
\end{equation}

Компоненты приложенных сил $F_m$ используют аналогичную схему
глобальной нумерации.

Надстрочным индексом $e$ будем обозначать величины, использующие локальную
нумерацию.

\subsection{Постановка задачи}
\label{sec:static-truss}

Рассмотрим схему применения метода конечных элементов для статического
анализа конструкции типа фермы, изображённой на \figref{fig:truss}.
Она состоит из двух конечных элементов \element{1} и \element{2}.

\begin{figure}[hb]
  \centering
  \begin{tikzpicture}
    \node [dot, label=below right:\node{1}] (1) at (0, 0) {};
    \node [dot, label=below right:\node{2}] (2) at ($ (1)+(0,4) $) {};
    \node [dot, label=below right:\node{3}] (3) at ($ (2)+(4,0) $) {};
    \draw (1) -- (3) -- (2);

    \draw[thick] ($ (1) - (0, .5) $) -- ($ (2) + (0, .5) $);

    \begin{scope}[>=latex]
      \draw [->] (3) -- ++(1.5,0) node[below] {$F_5$};
      \draw [->] (3) -- ++(0,1) node[left] {$F_6$};
    \end{scope}
  \end{tikzpicture}
  \caption{Ферма из трёх стержневых элементов}
  \label{fig:truss}
\end{figure}


Узлы \node{1} и \node{2} жёстко закреплены. Узлы \node{3} испытывает
внешнюю нагрузку. Задача состоит в нахождении смещений и реакций в
узлах модели.

Отметим, что в рассматриваемой модели смещения узлов вдоль локальной
оси $y$ ($v_1^e, v_2^e$) не связаны с жёсткостью элемента, поскольку
считается, что стержень передаёт только нагрузку, направленную вдоль
его локальной оси $x$.

Применение метода конечных элементов к данной конструкции заключается
в объединении условий равновесия \eqref{eq:bar-equil} каждого элемента
в общую систему уравнений, связывающую вектора узловых перемещений
$\matr{U}$ и нагрузок $\matr{F}$.

Рассмотрим порядок вывода системы уравнений. Пример расчёта с
использованием \fem{} для представленной конструкции приведены в
разделе \ref{sec:python-static}.

Более общий вывод с использование вариационных принципов дан в
\cite{bathe96} и \cite{zienkiewicz00}.

\subsection{Преобразование уравнений равновесия}

Для начала перепишем уравнение \eqref{eq:bar-equil}:
\begin{equation}
  \label{eq:bar-element-equil}
  \matr{k^e}\matr{u^e}=
  \begin{bmatrix}
    \phm{}k^e&& \mm k^e\\
    -k^e && k^e    
  \end{bmatrix}
  \begin{bmatrix}
    u_1^e\\
    u_2^e
  \end{bmatrix}
  =
  \begin{bmatrix}
    f_1^e\\
    f_2^e
  \end{bmatrix}=\matr{f^e}
\end{equation}
в глобальных координатах так, чтобы оно приняло следующий вид:
\begin{equation}
  \label{eq:target-equil}
  \matr{K^e}\matr{U^e}=\matr{K^e}
  \begin{bmatrix}
    U_1^e\\
    U_2^e\\
    U_3^e\\
    U_4^e
  \end{bmatrix}
  =
  \begin{bmatrix}
    F_1^e\\
    F_2^e\\
    F_3^e\\
    F_4^e
  \end{bmatrix}=\matr{F^e}.
\end{equation}

Для этого используем введённую в разделе \ref{sec:notation} матрицу
$R_0$, принимая во внимание отсутствие компонент $v_1^e, v_2^e$ в
уравнении равновесия. Определим матрицу $R$ следующим образом:
\begin{equation}
  \label{eq:short-rotmatrix}
  \matr{R} =
  \begin{bmatrix}
    \cos\theta & \sin\theta & 0 & 0 \\
    0 & 0 & \cos\theta & \sin\theta \\
  \end{bmatrix},
\end{equation}
так что с учётом \eqref{eq:glob-loc} получаем:
\begin{equation}
  \label{eq:short-glob-loc}
  \matr{u^e} = \matr{R}\matr{U^e}.
\end{equation}

Подстановка \eqref{eq:short-glob-loc} в \eqref{eq:bar-element-equil} даёт
\begin{equation*}
  \matr{k^e}\matr{R}\matr{U^e}=\matr{f^e}
\end{equation*}

Поскольку локальные координаты узловых нагрузок $f_1^e, f_2^e$
подчиняются соотношениям, подобным \eqref{eq:short-glob-loc}, переход
к вектору $\matr{F^e}$ в правой части \eqref{eq:bar-element-equil}
может быть осуществлён с помощью умножения на $\matr{R}^T$:
\begin{equation}
  \label{eq:bar-global-equil}
  \matr{R}^T\matr{k^e}\matr{R}\matr{U^e}=\matr{R}^T\matr{f^e} = \matr{F^e}.
\end{equation}

Сопоставляя последнее уравнение с \eqref{eq:target-equil}, заключаем,
что
\begin{equation}
  \label{eq:bar-global-stiffness}
  \matr{K^e} = \matr{R}^T\matr{k^e}\matr{R}.
\end{equation}

\subsection{Глобальная матрица жёсткости}

Дальнейшая задача состоит в объединении соотношений
\eqref{eq:target-equil} для каждого элемента в общую систему уравнений
\begin{equation}
  \label{eq:general-system}
  \matr{K}\matr{U}=\matr{F}
\end{equation}
с использованием глобальной нумерации, введённой в разделе
\ref{sec:notation}. Необходимо определить, каким образом элементы
матриц $\matr{K^e}$ объединяются в матрицу $\matr{K}$.

Воспользуемся определением произведения матриц и запишем выражение для
$m$-го элемента вектора $\matr{F}$ из \eqref{eq:general-system}:
\begin{equation*}
  F_m = \sum_{i=1}^{2N}{K_{mi}U_i}.
\end{equation*}

Формально заменим индексы $m \to L_m^{[n]}, i \to L_i^{[n]}$:
\begin{equation}
  F_{L_m^{[n]}} = \sum_{i=1}^{2N}{K_{{L_m^{[n]}},{L_i^{[n]}}}{U_{L_i^{[n]}}}}.
\end{equation}

В то же время, из \eqref{eq:target-equil} имеем:
\begin{equation}
  \label{eq:target-equil-prod}
  {F_m^e}^{[n]} = \sum_{i=1}^4{{K_{mi}^e}^{[n]}{U_i^e}^{[n]}},
\end{equation}
где с учётом \eqref{eq:num-vector-transform} перейдём к глобальным
индексам:
\begin{align*}
  {F_m^e}^{[n]} &\to F_{L_m^{[n]}},\\
  {U_i^e}^{[n]} &\to U_{L_i^{[n]}}.
\end{align*}
Но поскольку $\forall n$, $l\colon U_{L^{[n]}} \to {U^e}^{[n]}$ не
является инъекцией (за счёт соединений элементов индекс одного и того
же узла может встречаться в векторах связности разных элементов), для
правомерности перехода необходимо учесть все конечные элементы, в
результате чего получается основное соотношение, определяющее элементы
глобальной матрицы жёсткости:
\begin{equation}
  \label{eq:global-stiffness}
  K_{L_m^{[n]},{L_i^{[n]}}} = \sum_n{{K_{mi}^e}^{[n]}}
\end{equation}

Последнее равенство, вообще говоря, не позволяет прямым образом
определить произвольный элемент матрицы $K$. На практике алгоритм
построения $K$ реализуется итерационно с помощью перебора всех
конечных элементов модели с постепенным заполнением $K$.

\subsection{Решение}

После составления системы \eqref{eq:general-system} необходимо
приложить граничные условия начальных узловых смещений, определяющие
$2M$ из компонент вектора $\matr{U}$, и внешние нагрузки, задающие
$2N-2M$ координат $\matr{F}$.

В результате переупорядочивания строк система уравнений принимает вид:
\begin{equation*}
  \label{eq:system-partitioned}
  \begin{bmatrix}
    K_{cc}& K_{ca} \\
    K_{ac}& K_{aa}
  \end{bmatrix}
  \begin{bmatrix}
    U_c\\
    U_a\\
  \end{bmatrix}
  =
  \begin{bmatrix}
    F_c\\
    F_a
  \end{bmatrix},
\end{equation*}
где $\matr{U_c}$ — вектор начальных смещений, а $\matr{F_a}$ — вектор
внешних нагрузок.

Вектор неизвестных смещений $\matr{U_a}$ может быть найден из соотношения
\begin{equation}
  \matr{U_a} = \matr{K_{aa}}^{-1}(\matr{F_a} - \matr{K_{ac}}\matr{U_c})
\end{equation},
которое при распространённом в статике случае нулевого вектора
$\matr{U_c}$ имеет форму
\begin{equation}
  \label{eq:displacements}
  \matr{U_a} = \matr{K_{aa}}^{-1}\matr{F_a}
\end{equation}

После определения $\matr{U_a}$ вектор реакций в конструкции может быть
найден как $\matr{F_c} = \matr{K_{cc}}\matr{U_c} +
\matr{K_{ca}}\matr{U_a}$.

При известных узловых смещениях, деформации и напряжения определяются
с помощью соотношения \eqref{eq:bar-strain}.

\clearpage
\section{Динамическая задача}

При рассмотрении \emph{динамических} задач во внимание принимают
инерционные эффекты, возникающие в результате наличия массы у
элементов.

\subsection{Общий вид системы}

Рассмотрим свободные колебания системы, изображённой на
\figref{fig:truss2}.

\begin{figure}[hb]
  \centering
  \begin{tikzpicture}
    \node [dot, label=below right:\node{1}] (1) at (0, 0) {};
    \node [dot, label=above right:\node{2}] (2) at ($ (1)+(0,4) $) {};
    \node [dot, label=below right:\node{3}] (3) at ($ (1)+(4,0) $) {};
    \node [dot, label=above right:\node{4}] (4) at ($ (3)+(0,4) $) {};
    \node [dot, label=below right:\node{5}] (5) at ($ (3)+(4,0) $) {};
    \node [dot, label=above right:\node{6}] (6) at ($ (3)+(4,4) $) {};

    \draw (1) -- (3) -- (5) -- (6) -- (4) -- (2);
    \draw (1) -- (4) -- (5);
    \draw (3) -- (4);

    \draw[thick] ($ (1) - (0, .5) $) -- ($ (2) + (0, .5) $);
  \end{tikzpicture}
  \caption{Ферма из восьми стержневых элементов}
  \label{fig:truss2}
\end{figure}


Учитывая в соответствии с принципом Даламбера инерцию для записи
условий равновесия элементов системы, можно получить систему вида
\begin{equation}
  \label{eq:dynamic-system}
  \matr{M}\matr{\ddot{U}} + \matr{K}\matr{U} = \matr{0},
\end{equation}
которая представляет собой системы обыкновенных дифференциальных
уравнений второго порядка.

$\matr{M}$ — \neword{матрица масс} системы, которая, подобно матрице
жёсткости, формируется из элементных матриц масс.

Для стержневого элемента с плотностью $\rho$ матрица масс имеет вид:

\begin{equation}
  \label{eq:bar-mass}
  \matr{m^e} = \frac{\rho AL}{6}
  \begin{bmatrix}
    2&0&1&0\\
    0&2&0&1\\
    1&0&2&0\\
    0&1&0&2\\
  \end{bmatrix}
\end{equation}

\subsection{Собственные значения}

Решение \eqref{eq:dynamic-system} имеет вид
\begin{equation*}
U_i (t) = A_i \sin(\omega t + \phi),
\end{equation*}
где $A_i$ — амплитуда колебаний, а $\omega$ — их частота. Подстановка
решения в \eqref{eq:dynamic-system} приводит к уравнению
\begin{equation*}
  -\omega^2 \matr{M} + \matr{K} = 0,
\end{equation*}
которое, как известно, имеет нетривиально решение только в случае
нулевого детерминанта:
\begin{equation}
  \label{eq:eigenproblem}
  \det(\matr{K}-\omega^2\matr{M}) = 0
\end{equation}

Соотношение \eqref{eq:eigenproblem} представляет собой
\neword{частотное уравнение} системы, позволяющее определить
\neword{собственные частоты} ${\omega_i}$ динамической системы. Анализ
собственных частот важен при исследовании возможной реакции системы на
внешнее воздействие (вынуждающие колебания, которые при резонансе с
собственными частотами могут привести к разрушению системы).

В разделе \ref{sec:python-dynamic} представлены численные результаты
расчёта собственных частот фермы с \figref{fig:truss2}.

\section{Примеры и реализация метода}
\label{sec:python}

В ходе работы был создан прототип реализации \fem{} на языке Python.

При реализации метода конечных элементов необходимо учитывать, что
получающаяся глобальная матрица жёсткости обладает сильной
разреженностью и может быть плохо обусловлена.

\input{fem.tex}

\section{Выводы и результаты}

В результате работы осуществлено близкое знакомство с методом конечных
элементов, разработан прототип компьютерной реализации метода для
решения статических и динамических задач.

Заложены основы для дальнейшей работы в данном поле.

Среди направлений дальнейшего совершенствования реализации метода
необходимо выделить:
\begin{itemize}
\item Исследование эффективных методов представления разреженных
  матриц в памяти компьютера;

\item Разработка алгоритмов, хорошо работающих с большими разрешенными
  матрицами;

\item Использование параллельных вычислений при реализации (например,
  для перемножения матриц).
\end{itemize}

\clearpage
\appendix
\section{Информация о документе}

Данный документ был подготовлен с использованием \LaTeX{}. Код
представлен с использованием \program{noweb}. Иллюстрации были созданы
с помощью инструментов PGF и Ti$k$Z.

Представленная работа выполнена в рамках программы седьмого семестра
обучения по специальности «Вычислительная математика и математическая
физика» в МГТУ им. Н. Э. Баумана.

Дата компиляции настоящего документа: \today

\bibliographystyle{gost71s}
\bibliography{paper}

\end{document}
